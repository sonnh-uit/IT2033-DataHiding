\chapter*{\centering\Large{Tóm tắt đề tài}}
\addcontentsline{toc}{chapter}{Tóm tắt đề tài}


Sự phát triển của công nghệ mạng và truyền thông trong kỷ nguyên hiện đại đã làm tăng tốc độ truyền tải lên gấp hàng ngàn lần. Dữ liệu truyền tài trên mạng máy tính luân chuyển liên tục, đòi hỏi sự an toàn cho chúng. 

An toàn thông tin trong lĩnh vực này chia thành mã hóa thông tin và ẩn giấu thông tin. Mã hóa thông tin chuyển đổi những dữ liệu bí mật thành loại dữ liệu khác mà kẻ tấn công không thể đọc được nó. Tuy vậy, dữ liệu mã hóa trở thành ốc đảo giữa sa mạc, gây sự chú ý rất lớn từ kẻ tấn công. 

Do đó, một kỹ thuật khác nhằm bảo vệ dữ liệu là che giấu chính sự tồn tại của bí mật đó trước kẻ tấn công. Các dữ liệu được nhúng và gần như tàng hình trước kẻ xấu, và như vậy ít gây chú ý hơn. Trong lĩnh vực này, chia thành hai nội dung với những mục đích khác nhau. Trong khi kỹ thuật giấu tin ẩn vào dữ liệu được thực hiện nhằm mục đích bảo vệ sự bí mật của "dữ liệu được giấu" thì kỹ thuật watermark lại có mục đích bảo vệ chính dữ liệu đó. Với khả năng rút trích dữ liệu được giấu từ phiên bản số, ta dễ dàng chứng minh được tác quyền với nó \cite{subhedar2014current}.



