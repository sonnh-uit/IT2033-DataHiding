\chapter{LSB steganography using improved 1D chaotic map}
\label{chapter1}
% \section{Ẩn thông tin và các đặc điểm của nó}
% Ẩn giấu thông tin là phương pháp đang được sử dụng để  bảo mật và bí mật cho việc trao đổi dữ liệu. Thay vì dựa vào việc mã hóa thông điệp để bảo vệ nó khỏi sự xâm nhập, ẩn giấu thông tin đặt mục tiêu vào việc nhúng các thông tin nhạy cảm - từ các tệp tin, tin nhắn, hình ảnh, âm thanh cho đến video - vào bên trong các tệp tin gốc khác, có thể thuộc cùng một loại hoặc khác loại. Quá trình này được thực hiện một cách khéo léo để đảm bảo rằng dữ liệu ẩn được bảo vệ chặt chẽ và không thể dễ dàng nhận biết bởi bất kỳ ai ngoài các bên liên quan, chẳng hạn như người gửi và người nhận.

% Trong lĩnh vực bảo mật thông tin, mật mã thường được sử dụng để mã hóa và giải mã thông điệp, tạo ra một tầng bảo vệ vững chắc. Ẩn giấu thông tin tập trung vào việc chèn thông tin bí mật mà không gây ra bất kỳ thay đổi nào trong dữ liệu gốc. Điều này đặc biệt hữu ích khi cần duy trì tính nguyên vẹn của dữ liệu gốc mà vẫn muốn lưu trữ thông tin bí mật.

% Tính không thể nhận thấy là một đặc trưng quan trọng của ẩn giấu thông tin, được gọi là tính trong suốt hoặc hiệu suất chống phát hiện. Khả năng này đảm bảo rằng dữ liệu ẩn không dễ dàng bị phát hiện bởi những phương pháp kiểm tra thông thường. Để nâng cao tính trong suốt của kỹ thuật này, có thể thực hiện các cải tiến trong phương pháp ẩn giấu thông tin hoặc tăng cường mối quan hệ giữa thông tin bí mật và tệp chứa thông tin.

% \section{Ẩn thông tin dựa trên bit ít quan trọng nhất (LSB Steganography)}

Least Significant Bit (LSB) Steganography là một kỹ thuật trong lĩnh vực ẩn giấu thông tin, được sử dụng để nhúng thông tin bí mật vào trong một tập tin đa phương tiện, chẳng hạn như hình ảnh, âm thanh hoặc video. Kỹ thuật này tận dụng tính chất của các bit ít quan trọng nhất (Least Significant Bits) trong các dữ liệu số như điểm ảnh, mẫu âm thanh hoặc khung hình video. LSB Steganography cho phép nhúng thông tin ẩn vào những bit ít quan trọng này mà không gây ra sự thay đổi đáng kể cho dữ liệu gốc.

Nguyên tắc hoạt động của LSB Steganography rất đơn giản: trong một tập tin đa phương tiện, các dữ liệu như điểm ảnh thường được biểu diễn bằng các chuỗi bit. Các bit ít quan trọng nhất thường có giá trị thấp hơn và có xu hướng thay đổi ít ảnh hưởng đến hình ảnh hoặc âm thanh tổng thể. Điều này tạo ra một cơ hội tốt để thay thế những bit này bằng các bit của thông tin ẩn, giữ nguyên tính nguyên vẹn của dữ liệu gốc mà vẫn chèn thông tin bí mật vào.

% Một ví dụ cụ thể có thể là nhúng một chuỗi văn bản thông điệp vào một hình ảnh bằng cách thay thế các bit ít quan trọng nhất của các điểm ảnh trong hình ảnh đó. Khi xem hình ảnh, sự thay đổi này thường không dễ dàng bị nhận thấy bởi mắt người. Tuy nhiên, người nhận có thể sử dụng một thuật toán tương ứng để trích xuất thông điệp ẩn ra khỏi hình ảnh.

% Mặc dù LSB Steganography đơn giản và dễ triển khai, nó vẫn có nhược điểm. Việc nhúng thông tin quá nhiều có thể làm thay đổi tới mức có thể nhận thấy được trong dữ liệu đầu ra. Ngoài ra, nếu người tấn công biết rằng một hình ảnh hoặc tập tin âm thanh đã được sử dụng để nhúng thông tin, họ có thể dễ dàng tìm ra thông điệp ẩn.

% Image steganography được phân loại thành ẩn trong space-domain và conversion-domain.
\section{Chaotic map}
Chaotic map được chia thành map một chiều (1D) và đa chiều  (multi-dimensional). Nó thường được sử dụng trong mã hóa vì nó có tính ngẫu nhiên của chuỗi hỗn loạn được tạo. Mặc dù phiên bản 1D có một số nhược điểm nhưng chúng được sử dụng rộng rãi
do cấu trúc đơn giản và chi phí tính toán thấp, các nghiên cứu đang được thực hiện
để cải thiện hiệu suất của nó.

\subsection{Logistic và sine map hiện tại}

Logistic map hiện tại có thể biểu diễn đơn giản như sau:
\begin{equation}
\label{eq:logistic_map}
    x_{n+1} = u \times x_n \times (1 - x_n)
\end{equation}
trong đó $u \in [0,4\}$ là tham số điều khiển của hàm chaotic và $x_0 \in [0,1\}$ là giá trị ban đầu của nó.

Sine map hiện tại được thể hiện như sau:
\begin{equation}
\label{eq:sine_map}
x_{n + 1} = r \times \sin(\pi \times x_n)
\end{equation}
trong đó $r \in (0,1]$ là tham số điều khiển của sine map. Sine map và logistic map thể hiện trong \textbf{Hình \ref{fig:chap1-logistic_sine_map}}

\begin{figure}
    \centering
    \includegraphics[scale=0.7]{graphics/chapter-1/chap1-logistic_sine_map.png}
    \caption{Logistic map (a) và sine map (b)}
    \label{fig:chap1-logistic_sine_map}
\end{figure}

\subsection{Improved 1D chaotic system model}

Hàm 1D chaotic được cải tiến được thể hiện như sau:

\begin{equation}
\label{eq:improved-1d-chaotic}
\begin{multlined}
f_{ic}(u,x_{n+1},k) = mod(f_c(u,x_n) \times g(k),1). \\
where g(k) = 2^k, 9 \leq k \leq 16.
\end{multlined}
\end{equation}

Tham số hệ thống $u$ vẫn nằm trong đoạn $(0,4]$ và có thể mở rộng thêm. $f_c(u,x_n)$ là hàm 1D chaotic map hiện tại, $f_{ic}(u,x_{n+1},k)$ là hàm đã được cải tiến của mô hình đề xuất. $k=12$ là giá trị tốt nhất (sau khi đã thực nghiệm).

\subsection{Improved 1D chaotic map and performance evaluation}

Dựa trên những cải tiến của hàm chaotic tại \ref{eq:improved-1d-chaotic}, hàm logistic map được thể hiện như sau:
\begin{equation}
\label{eq:improved_logistic_map}
x_{n+1} = mod(u \times x_n \times (1 - x_n ) \times 2^{12}, 1)
\end{equation}
và hàm sine map được cải tiến như sau:
\begin{equation}
\label{eq:improved_sine_map}
x_{n+1} = mod( u \times \sin(\pi \times x_n) \times 2^{12},1)
\end{equation}

\textbf{Hình \ref{fig:chap1-improved_map}} mô tả phân bố của các hàm logistic và sine sau khi cải tiến hàm chaotic.

\begin{figure}
    \centering
    \includegraphics[scale=0.7]{graphics/chapter-1/chap1-improved_map.png}
    \caption{Distribubtions của hai hàm logisctic map (c) và sine map}
    \label{fig:chap1-improved_map}
\end{figure}

Số mũ Lyapunov là một chỉ số để đánh giá  hiệu suất của chaotic map và có các đặc tính hỗn loạn tốt khi giá trị lớn hơn 0. Số mũ Lyapunov của chaotic map hiện tại và của phiên bản cải thiện được thể hiện trong \textbf{Hình \ref{fig:chap1-lyapunov}}. 



\begin{figure}
    \centering
    \includegraphics[scale=0.7]{graphics/chapter-1/chap1-lyapunov.png}
    \caption{Biểu đồ số mũ Lyapunov logistic map (a), sine map (b) và phiên bản cải tiến của nó}
    \label{fig:chap1-lyapunov}
\end{figure}

Ngoài ra, entropy là chỉ số đánh giá hiệu suất của chaotic map, là một chỉ số để đánh giá sự không ổn định của các giá trị ngẫu nhiên, có nghĩa là nó có tối đa giá trị khi tất cả các tín hiệu được phân phối ngẫu nhiên và giá trị càng gần với 8 thì đặc tính phân phối càng tốt.
Hàm entropy, được thể hiện trong công thức sau
\begin{equation}
\label{eq:entropy}
I(R) = \sum_{i=0}^{F - 1 } P(R=i) \times \log_2P(R = i)
\end{equation}
trong đó $P$ là hàm mật độ xác suất rời rạc. \textbf{Hình \ref{fig:chap1-entropy}} cho thấy sự tương phản giữa chỉ số entropy chaotic map hiện có và phiên bản được cải tiến. Nó cho thấy rằng chaotic map được cải thiện có các đặc điểm phân phối hoàn toàn vượt trội. Các kết quả thử nghiệm này cho thấy mô hình hệ 1D chaotic system được đề xuất là rất chính xác.

\begin{figure}
    \centering
    \includegraphics[scale=0.5]{graphics/chapter-1/chap1-entropy.png}
    \caption{Chỉ số entropy của hàm hiện tại (a) và bản cải ttiến}
    \label{fig:chap1-entropy}
\end{figure}

\section{Proposed LSB steganography algorithm}
\subsection{Embedding process}

Quá trình nhúng là một quá trình ẩn dữ liệu bí mật trong ảnh bìa và các tệp hình ảnh và văn bản có thể được sử dụng làm dữ liệu bí mật. Thuật toán như sau.

\begin{itemize}
    \item \textbf{Bước 1}: Đọc ảnh bìa và dữ liệu bí mật.
    Đọc ảnh bìa và lấy giá trị pixel trung bình của ảnh bìa theo phương trình sau
    \begin{equation}
    \label{eq:read_cover_secret}
    PM_c = \sum_{k=1}^c \sum_{i=1}^M \sum_{j=1}^n cp_{ij} \mathbin{/} (c \times M \times N)
    \end{equation}
    với $c=3$ là giá trị của bảng màu RGB, $M$ và $N$ là chiều cao và chiều rộng của ảnh bìa và $cp_{ij}$ là giá trị pixel của ảnh bìa. 
    Tiếp theo, dữ liệu bí mật được nhúng được đọc và chuyển đổi thành ma trận một chiều và giá trị trung bình của dữ liệu bí mật thu được theo phương trình sau.
    \begin{equation}
    \label{eq:secret_data}
    PM_s = \sum_{i=1}^{sLen} sd_i \mathbin{/} sLen
    \end{equation}
    với $sd_i$ là giá trị phần tử ma trận một chiều của dữ liệu bí mật và sLen là độ dài dữ liệu bí mật.
    \item \textbf{Bước 2:} Xác định các tham số ban đầu của chaotic map và tạo khóa.
    Từ giá trị $PM_c$ và $PM_s$ được tính tại \textbf{Bước 1}, các tham số ban đầu của chaotic map được tính toán như sau.
    \begin{equation}
    \label{eq:chaotic_x0}
    x'_0 = x_0 \times PM_c - floor(x_0 \times PM_c)
    \end{equation}
    \begin{equation}
    \label{eq:chaotic_u}
   u' = 4 \times (u \times PM_s - floor(u \times PM_s))
    \end{equation}
    với $u' \in [0,4], x'_0 \in [0,1]$
    Bằng cách đặt $x'_0$ và $u'$ được tính toán lại làm tham số hệ thống ban đầu của 1D chaotic map bản cải tiến, chúng ta có thể rút ra chuỗi $XM = \{x_1, x_2,...x-s \}$ có kích thước $s$ được tính theo phương trình \ref{eq:s_size_output} và $bc_{lsb}=4$ là số bit LSB của ảnh bìa.
    \begin{equation}
    \label{eq:s_size_output}
    s = (c \times M \times N \times bc_{lsb})
    \end{equation}
    \item \textbf{Bước 3:}Tính toán ma trận vị trí nhúng.
    Chúng ta có ma trận vị trí $P = \{p_1, p_2,...p_s \}$ bằng cách sắp xếp đầu ra của chaotic map $XM$ theo quy trình tại \textbf{Hình \ref{fig:chap1-position-matrix-generate}}
    Từ ma trận vị trí $P$, chúng ta tính toán được ma trận nhúng $EP = \{ep_1, ep_2,...ep_s \}$ với $ep_i = \{ p_{cp}(i), p_{row}(i), p_{col}(i)\}$ được tính theo các phương trình \ref{eq:embed_pcp}, \ref{eq:embed_prow}, \ref{eq:embed_pcol}
    \begin{equation}
    \label{eq:embed_pcp}
    p_{cp}(i) = floor((p(i) - 1) \mathbin{/}(M \times N)) + 1
    \end{equation}
    \begin{equation}
    \label{eq:embed_prow}
    p_{row}(i) = mod((p(i) -1), N) +1
    \end{equation}
    \begin{equation}
    \label{eq:embed_pcol}
    p_{col}(i) = mod((floor((p(i) -1) \mathbin{/} M),N) +1
    \end{equation}
    với $ 1 \leq p_{cp}(i) \leq c, 1 \leq p_{row}(i) \leq M, 1 \leq p_{col}(i) \leq N.$
    $p_{cp}(i)$là số bảng màu của vị trí nhúng dữ liệu bí mật, $p_{row}(i)$ là vị trí hàng được nhúng và $p_{col}(i)$ là vị trí cột.
    Phương trình \ref{eq:embed_pcp} - \ref{eq:embed_pcol} được tính toán lặp đi lặp lại theo $s$ lần và ma trận vị trí nhúng thu được cung cấp tất cả thông tin về các vị trí nhúng dữ liệu bí mật.
    \item \textbf{Bước 4}: Quy trình nhúng
    Đầu tiên tạo ma trận bit một chiều $sb = \{sb_1, sb_2,...sb_{sLen}\}$ của dữ liệu bí mật. Theo ma trận $EP$ được tính toán tại \textbf{Bước 3}, bốn bit của $sb$ ma trận sẽ được nhúng trong các bit LSB của ảnh bìa dưới dạng một đơn vị và quá trình này có thể được biểu diễn như sau.
    \begin{equation}
    \label{eq:embeđe_process}
    cp_{lsb}(p_{cp}(i), p_{row}(i), p_{col}(i)) = sb(i \times 4 -3 : i \times 4)
    \end{equation}
    với $1 \leq i \leq (2 \times sLen)$ và $cp_{lsb}$ là bit LSB của ảnh bìa.
\end{itemize}
\textbf{Bước 4} thực hiện $(2 \times sLen)$ lần tới chừng nào hình ảnh stego hoàn thành. Trong trường hợp dữ liệu bí mật được nhúng là một hình ảnh, cần phải chuyển đổi dữ liệu hình ảnh bí mật sang ma trận một chiều ở \textbf{Bước 1}
\begin{figure}
    \centering
    \includegraphics[scale=0.6]{graphics/chapter-1/chap1-proposed_algorithm.png}
    \caption{Lưu đồ tổng thể thuật toán đề xuất}
    \label{fig:enter-label}
\end{figure}

\begin{figure}
    \centering
    \includegraphics[scale=0.6]{graphics/chapter-1/chap1-position-matrix-generate.png}
    \caption{Sơ đồ sinh ma trận vị trí}
    \label{fig:chap1-position-matrix-generate}
\end{figure}

\subsection{Extracting process}
Ở giai đoạn này, trích xuất dữ liệu bí mật được chèn vào hình ảnh stego và nó có thể được coi là quá trình ngược lại với quy trình nhúng. $(x_0, u_0, k, PM_c, PM_s, sLen)$ là các khóa bí mật và quá trình trích xuất có thể được biểu diễn như sau.
\begin{equation}
\label{eq:extract_process}
sb(i \times 4 - 3 : i \times 4) = cp_{lsb}(p_{cp}(i), p_{row}(i), p_{col}(i))
\end{equation}
với $1 \leq i \leq (2 \times sLen)$. Ma trận bit thu được bằng cách lặp lại $(s \times sLen)$ lần được chuyển thành ma trận byte. Do đó, thông tin bí mật ban đầu thu được.
Trường hợp dữ liệu bí mật được nhúng là ảnh thì cần chuyển ma trận một chiều thu được thành ma trận hai chiều tương ứng với kích thước của ảnh mật.